\documentclass[sigconf]{acmart}

\input{format/i523}

\begin{document}
\title{Applications of Big Data Analytics in Public Policy Development and Evaluation}


\author{Nsikan Udoyen}
\orcid{}
\affiliation{%
  \institution{School of Informatics and Computing, Indiana University}
  \streetaddress{P.O. Box 1212}
  \city{Bloomington} 
  \state{IN} 
  \postcode{47408}
  \country{USA}}
\email{nudoyen@iu.edu}


% The default list of authors is too long for headers}
\renewcommand{\shortauthors}{N. Udoyen}


\begin{abstract}
This paper provides a sample of a \LaTeX\ document which conforms,
somewhat loosely, to the formatting guidelines for
ACM SIG Proceedings.
\end{abstract}

\keywords{Big Data, Edge Computing i523}


\maketitle



\section{Introduction}

The use of big data analytics has the potential to transform public policy by enabling policymakers to better understand public policy needs and craft laws that are more responsive them. Big data analytics also provide policymakers with tools to better assess the effectiveness of policies based on current and very detailed data. These tools can be used in a wide variety of policymaking scenarios, but because of the methods used to gather the data, their use also raises questions about ethical issues such as privacy, ownership of data, and whether or not the use of big data analytics, despite its potential advantages, should substitute grassroots public engagement and public involvement in policymaking.

\section{Big Data Analytics in Public Policy Development: Key Research Areas}
In a survey of big data applications in policymaking, \cite{giest_2017}, the author discusses benefits and barriers to big data in public policy, and reviews research that highlights three recurrent themes associated with research on big data in public policy:

\begin{itemize}
   \item Government's readiness to use big data analytics, and the view that infrastructure needs to be put in place for public policymakers to benefit from big data analytics
   \item The use of big data to support delivery of digital public services
   \item How big data is incorporated into policymaking, and the risks involved in ineffective incorporation of big data into the policymaking cycle
\end{itemize}

She concludes that though big data offers tangible benefits, there are significant barriers that must be overcome to successfully incorporate big data into public policymaking.

\section{Big Data Analytics in Public Policy Development: Applications}
\subsection{Public Housing}
In a paper on the application of big data methods to federal housing policy \cite{davidson_2017}, the author examines relevance to affordable housing efforts administered by the US Department of Housing and Urban Development. He lists more efficient targeting of resources, and improved collaboration between agencies, better decision-making as benefits of big data, but notes that the emphasis on quantifiable goals could redirect resources away from goals that are not easily quantified, thus further marginalizing groups that may already be vulnerable.

\subsection{Health Policy}
In their paper on big data analytics in healthcare \cite{alemayehu_2016}, the authors discuss several potential uses of big data in various aspects of clinical research that supports development of new drugs, and associated healthcare policies (primarily drug licensing and approval). They discuss the randomized controlled clinical trials that occur during the development of drugs, and suggest that big data methods and tools could enable data from observational studies, health records and other sources to be used to support data from clinical trials during drug development. They list several benefits and risks, and highlight the public policy challenges involved in approving drugs developed using data other than clinical trial results.

\subsection{Security}
In their paper on security policy \cite{graves_2016}, the authors list missing data, inaccurate data, and poorly interpreted data as sources of error that impact security policy. While their emphasis is on cybercrime, the breadth of their discussion incorporates national security concerns, such as data breaches at federal agencies, such as the Office of Personnel Management. While advocating a "science of cybersecurity", they suggest the development of statistical models that can be used with big data to assess the impact of cybercrimes, and advise allocation of more resources towards better collection of data that can be used with such models.

\subsection{Electronic Participation (e-Participation)}
In their paper on the use of big data to encourage public participation in policymaking \cite{bright_2016}, the authors review several attempts by various governments around the world to use digital platforms to solicit inputs from the public and active participation in policymaking. They list the common themes in e-Participation schemes as
\begin{itemize} 
   \item Access to the internet, and the specific tools used for e-Participation
   \item Changes to political processes prompted by technology
   \item Disappointment with failed electronic participation initiatives and its impact on participation in future projects
\end{itemize}
These themes are described by the authors as the basis for the interest in big data as a way of enabling public participation in policymaking. They discuss the challenges involved in expanding the scope of data beyond administrative statistics to include data from social media. They also note that data from social media is "passively contributed", in that members of the public do not typically post data online with the intent that it will be used to shape public policy. 

\section{Discussion}
The key themes highlighted in the work by (Geist) were revisited in all of the domain-specific papers reviewed in the preceding section. Several proposals on how to incorporate big data into public policymaking were discussed in the papers on affordable public housing and drug development. The use of big data to support delivery of digital services was discussed in the context of electronic participation in public policymaking. The ongoing questions of how prepared government is to incorporate big data methods into policymaking, and what needs to be done were illustrated the papers on security policy and affordable housing.
 
The issue of privacy has also been brought up \cite{stough_2014}, along with the ethical questions concerning the methods used to gather detailed data about individuals, what constitutes fair use of gathered data, and the ownership of the gathered data. Such questions remain open, and have not been definitively answered in research.


\section{Conclusion}
Put here an conclusion. Conclusions and abstracts must not have any
citations in the section.


\bibliographystyle{ACM-Reference-Format}
\bibliography{report} 



\end{document}
