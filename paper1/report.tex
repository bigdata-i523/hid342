\documentclass[sigconf]{acmart}

\usepackage{hyperref}

\usepackage{endfloat}
\renewcommand{\efloatseparator}{\mbox{}} % no new page between figures

\usepackage{booktabs} % For formal tables

\settopmatter{printacmref=false} % Removes citation information below abstract
\renewcommand\footnotetextcopyrightpermission[1]{} % removes footnote with conference information in first column
\pagestyle{plain} % removes running headers

\begin{document}
\title{Big Data in NCAA Football}


\author{Nsikan Udoyen}
\orcid{1234-5678-9012}
\affiliation{%
  \institution{School of Informatics and Computing, Indiana University}
  \streetaddress{P.O. Box 1212}
  \city{Dublin} 
  \state{Indiana} 
  \postcode{43017-6221}
}
\email{nudoyen@iu.edu}

% The default list of authors is too long for headers}
\renewcommand{\shortauthors}{N. Udoyen et al.}


\begin{abstract}
This paper provides an overview of applications of big data in NCAA football.
\end{abstract}

\keywords{i523}


\maketitle

\section{Introduction}

National Collegiate Athletics Association (NCAA) football is one of the most widely watched sports in the United States. 
The size of the fan base and the profits that can be derived from televised games incentivizes universities and other interested parties 
to invest in the application of big data analytics and data science methods in general to improve on-field outcomes by enabling better 
management of player well-being and performance. The purpose of this paper is to provide an overview of the use of data science in 
National Collegiate Athletics Association (NCAA) football. Recent research on the use of data science to improve various aspects of NCAA 
football will be surveyed, while current trends and their implications will be discussed. 


\begin{acks}

  The authors would like to thank 

\end{acks}

\bibliographystyle{ACM-Reference-Format}
\bibliography{report} 

\end{document}
